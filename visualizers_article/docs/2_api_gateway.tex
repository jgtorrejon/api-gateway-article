\section{Api Gateway}
Antes de hablar de las diferencias entre Kong y AWS API Gateway, es importante
entender qué es un API Gateway. Un API Gateway es un servidor que actúa como
un punto de entrada para una API. Es similar a un proxy inverso, pero con
funcionalidades adicionales. Un API Gateway puede realizar tareas como
autenticación, autorización, enrutamiento, transformación de datos, monitoreo
y administración de versiones. También puede proporcionar una capa de seguridad
adicional al ocultar la estructura de la red interna de una
organización. \cite{red-hat-api-gateway}

\subsection{Funciones}
La tecnología API Gateway ofrece una serie de ventajas, como la gestión
eficiente de las solicitudes entrantes, que las enruta fácilmente a los
servicios backend pertinentes. Además, puede traducir automáticamente
los protocolos para que los clientes puedan interactuar con el servicio
sin esfuerzo. \cite{kong-what-is-api-gateway}

\subsubsection{Baja latencia :}
Al distribuir las solicitudes entrantes y descargar tareas comunes como
la terminación SSL y el almacenamiento en caché, las puertas de enlace
de API optimizan el enrutamiento del tráfico y el equilibrio de carga
entre los servicios de backend para garantizar un rendimiento y una
utilización de recursos óptimos. De este modo, las puertas de enlace de
API minimizan la carga del servidor y el uso del ancho de banda, lo que
reduce la necesidad de capacidad adicional del servidor y del ancho de
banda de la red y mejora la experiencia del usuario.

\subsubsection{Gestión del tráfico :}
Las puertas de enlace API limitan y gestionan el tráfico a través de
diversos mecanismos diseñados para controlar la velocidad y el volumen
de las solicitudes entrantes y garantizar un rendimiento y una
utilización de recursos óptimos.

\begin{itemize}
  \item {\textbf{Las políticas de limitación de velocidad}}
  \item {\textbf{Las políticas de limitación de solicitudes}}
  \item {\textbf{Las políticas de control de concurrencia}}
  \item {\textbf{Las políticas de interrupción de circuitos}}
  \item {\textbf{El equilibrio de carga dinámico}}
\end{itemize}

\subsubsection{Aprovechamiento de la infraestructura de red global :}
Las puertas de enlace API pueden escalar dinámicamente los recursos de
infraestructura en respuesta a los cambios en los patrones de tráfico
y las demandas de carga de trabajo. De esta manera, las puertas de
enlace API ayudan a las empresas a optimizar el uso de los recursos
y minimizar los costos de infraestructura, lo que garantiza que
solo paguen por los recursos que realmente utilizan.

\subsubsection{Rentabilidad}
Las puertas de enlace de API desempeñan un papel en la gestión de
la rentabilidad de la entrega de aplicaciones y la integración de
API al proporcionar una plataforma centralizada para gestionar el
tráfico de API, aplicar políticas de seguridad, implementar reglas
de gestión del tráfico y facilitar la integración con los servicios
de backend. Las puertas de enlace de API también permiten el 
consumo escalonado de servicios para mantener la rentabilidad. Los
diferentes tipos de API pueden afectar la rentabilidad de una
aplicación de varias maneras.

\begin{itemize}
  \item {\textbf{Flexibilidad}}
  \item {\textbf{Infraestructura}}
  \item {\textbf{Escalabilidad}}
\end{itemize}